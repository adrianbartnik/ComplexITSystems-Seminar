\documentclass[conference]{IEEEtran}

\usepackage[english]{babel} 
\usepackage[utf8]{inputenc}
\usepackage[T1]{fontenc}
\usepackage{cite}
\usepackage{url}
\usepackage{hyphenat} % For linebreaks
\usepackage{hyperref}
\usepackage{geometry}

% correct bad hyphenation here
% \hyphenation{op-tical net-works semi-conduc-tor}


\begin{document}

\title{Stream processing with Twitter Heron}

\author{\IEEEauthorblockN{Adrian Bartnik}
\IEEEauthorblockA{Technische Universität Berlin\\
Email: bartnik@campus.tu-berlin.de}}

\maketitle

\begin{abstract}
This paper gives an overview about the requirements and the demands of a real-time stream processing platform.
As an example, it presents the architecture and design decisions of Apache Storm, its successor Twitter Heron and Google`s MillWheel. % Asumptions?
It compares each streaming platform in terms of prior assumptions, scalability and extensibility.
Finally, it presents a summary with the advantages and disadvantages of each platform and gives an outlook for future use cases.

\end{abstract}

\section{Introduction}

In recent years, stream processing has become a new companion to the family of big data buzzwords.
This paper presents the challenges that come with it and a state-of-the-art technology overview.

Twitter is a real-time messaging service with XXX millions users each month.
Their record in most processed tweets per minute was 618,725, which demonstrates the sheer scale on which the platform has to operate.

\subsection{Real-time Stream Processing}

Stream processing has many aspects, that have to be considered independently.

\subsection{Requirements of a Stream Processing Platform}

Modern requirements for a real-time stream processing platform.

\cite{The8Requirements}
\cite{ElasticScalingStreamProcessing}
\cite{OnlyOneLook}
\cite{YARN}
\cite{ScalableDistributedStreamProcessing}

Describe structure of the paper.
First, it describes the requirements of a stream processing platform.
After that, it presents the 3 streaming platforms Apache Storm, Twitter Heron and Google MillWheel in detail.

\section{Related Work}

\cite{InfoQGameChanger}

\section{Apache Storm}

\cite{StormTwitter}

\section{Twitter Heron}

\subsection{Evolution from Apache Storm}

\subsection{Twitter Heron Architecture}

% \section{Google MillWheel}

% \cite{Millwheel}

\section{Evaluation}

\cite{TwitterHeronBlog}

\cite{TwitterHeron}


\section{Conclusion}

The conclusion goes here.


% References

\bibliographystyle{plain} %Choose a bibliograhpic style
\bibliography{Bibliography}

\end{document}