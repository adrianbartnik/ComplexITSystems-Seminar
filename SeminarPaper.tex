\documentclass[conference]{IEEEtran}

\usepackage[english]{babel} 
\usepackage[utf8]{inputenc}
\usepackage[T1]{fontenc}
\usepackage{cite}
\usepackage{url}
\usepackage{hyphenat} % For linebreaks
\usepackage{hyperref}
\usepackage{geometry}

% correct bad hyphenation here
% \hyphenation{op-tical net-works semi-conduc-tor}


\begin{document}

\title{Stream processing with Twitter Heron}

\author{\IEEEauthorblockN{Adrian Bartnik}
\IEEEauthorblockA{Technische Universität Berlin\\
Email: bartnik@campus.tu-berlin.de}}

\maketitle

\begin{abstract}
This paper gives an overview about the requirements and the demands of a real-time stream processing platform.
As an example, it presents the architecture and design decisions of Apache Storm and its successor Twitter Heron.
It gives compares both approaches and gives and an outlook for future.

\end{abstract}

\section{Introduction}

In recent years, stream processing has become a new companion to the family of big data buzzwords.
This paper presents the challenges that come with it and a state of the art technology overview.

Twitter is a real-time messaging service with XXX millions users each month.

Stream processing has many aspects, that have to be considered independently.

Describe structure of the paper.
First, it describes the requirements of a stream processing platform.

\section{Requirements of a Stream Processing Platform}

\section{Background}
\subsection{Apache Storm Architecture}
\subsection{Twitter Heron Architecture}

\section{Evaluation}

Most of the academic staff in the School of Computer Science and Software Engineering have publications in their field of expertise or research. 
The majority of the publications appear in journals or proceedings of conferences. 
Articles appearing in journals may be written by a single author.

\cite{YARN}
\cite{TwitterHeronBlog}
\cite{InfoQGameChanger}
\cite{ElasticScalingStreamProcessing}
\cite{OnlyOneLook}
\cite{TwitterHeron}
\cite{The8Requirements}
\cite{StormTwitter}
\cite{ScalableDistributedStreamProcessing}


\section{Conclusion}

The conclusion goes here.


% References

\bibliographystyle{plain} %Choose a bibliograhpic style
\bibliography{Bibliography}

\end{document}

\end{document}