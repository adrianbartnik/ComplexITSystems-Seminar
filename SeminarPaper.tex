\documentclass[conference]{IEEEtran}

\usepackage[english]{babel} 
\usepackage[utf8]{inputenc}
\usepackage[T1]{fontenc}
\usepackage{cite}
\usepackage{url}
\usepackage{hyphenat} % For linebreaks
\usepackage{hyperref}
\usepackage{geometry}

% correct bad hyphenation here
% \hyphenation{op-tical net-works semi-conduc-tor}


\begin{document}

\title{Stream processing with Twitter Heron}

\author{\IEEEauthorblockN{Adrian Bartnik}
\IEEEauthorblockA{Technische Universität Berlin\\
Email: bartnik@campus.tu-berlin.de}}

\maketitle

\begin{abstract}
This paper gives an overview about the requirements and the demands of a real-time stream processing platform.
As an example, it presents the architecture and design decisions of Apache Storm, its successor Twitter Heron and Google`s MillWheel. % Asumptions?
It compares each streaming platform in terms of prior assumptions, scalability and extensibility.
Finally, it presents a summary with the advantages and disadvantages of each platform and gives an outlook for future use cases.

\end{abstract}

\section{Introduction}
\label{sec:Introduction}

In recent years, stream processing has become a new companion to the family of big data buzzwords.
It describes the paradigm of processing incoming high-volume data stream, where the data can come from various sources.
Scenarios are micro transactions at a stock exchange, sensor measurements from small independent devices in the upcoming area of the Internet of Things or analytic data coming in from millions of mobile phones.

Most major Internet-scale companies either have their own proprietary stream processing platform or use some open-source alternative.
One prominent example is Twitter, a real-time messaging service with over 320 million active users and over 1 billion unique site visits each month.
Twitter uses stream processing in various places, for example to compute the trending topics, a small section on their website showing the currently most used, and therefore most relevant keywords by their users.
In 2011 Twitter started off developing its own stream processing platform Twitter Storm after acquiring the technology from the company BackType.
Shortly afterwards, they also made the whole project available under open source and it became an Apache top-level project.

But, using Storm at its scale was becoming increasingly challenging due to issues related to scalability, debug-ability, manageability, and efficient sharing of cluster resources with other data services.
The company therefore started to develop a new system, Twitter Heron, which should fix their problems and also be able to scale for the future.
Twitter Heron is now the de facto stream data processing platform inside Twitter.

The rest of the paper is organized as follows: The following section, Section \ref{sec:Background}, describes the characteristics of real-time stream processing and the requirements of a stream processing platform.
Section \ref{sec:RelatedWork} describes other platforms and related work.
Section \ref{sec:ApacheStorm} and \ref{sec:TwitterHeron} describe Apache Strom and Titter Heron respectively, where the later also describes the evolution between the two platform.
The evaluation in section \ref{sec:Evaluation} compares both approaches.
Finally, section \ref{sec:Conclusion} contains the conclusion.

\section{Background}
\label{sec:Background}

\subsection{Real-time Stream Processing}
\label{sec:RealTimeStreamProcessing}

Stream processing has many aspects, that have to be considered independently.

\subsection{Requirements of a Stream Processing Platform}
\label{sec:RequirementsOfAStreamProcessingPlatform}

Modern requirements for a real-time stream processing platform.

\cite{The8Requirements}
\cite{ElasticScalingStreamProcessing}
\cite{OnlyOneLook}
\cite{YARN}
\cite{ScalableDistributedStreamProcessing}

Describe structure of the paper.
First, it describes the requirements of a stream processing platform.
After that, it presents the 3 streaming platforms Apache Storm, Twitter Heron and Google MillWheel in detail.

\section{Related Work}
\label{sec:RelatedWork}

\cite{InfoQGameChanger}

\section{Apache Storm}
\label{sec:ApacheStorm}

\cite{StormTwitter}

\section{Twitter Heron}
\label{sec:TwitterHeron}

\subsection{Evolution from Apache Storm}
\label{sec:EvolutionFromApacheStorm}

\subsection{Twitter Heron Architecture}
\label{sec:TwitterHeronArchitecture}

% \section{Google MillWheel}

% \cite{Millwheel}

\section{Evaluation}
\label{sec:Evaluation}

\cite{TwitterHeronBlog}

\cite{TwitterHeron}


\section{Conclusion}
\label{sec:Conclusion}

The conclusion goes here.


% References

\bibliographystyle{plain} %Choose a bibliograhpic style
\bibliography{Bibliography}

\end{document}